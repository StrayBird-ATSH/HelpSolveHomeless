\documentclass[]{article}
\usepackage{lmodern}
\usepackage{amssymb,amsmath}
\usepackage{ifxetex,ifluatex}
\usepackage{fixltx2e} % provides \textsubscript
\ifnum 0\ifxetex 1\fi\ifluatex 1\fi=0 % if pdftex
  \usepackage[T1]{fontenc}
  \usepackage[utf8]{inputenc}
\else % if luatex or xelatex
  \ifxetex
    \usepackage{mathspec}
  \else
    \usepackage{fontspec}
  \fi
  \defaultfontfeatures{Ligatures=TeX,Scale=MatchLowercase}
\fi
% use upquote if available, for straight quotes in verbatim environments
\IfFileExists{upquote.sty}{\usepackage{upquote}}{}
% use microtype if available
\IfFileExists{microtype.sty}{%
\usepackage{microtype}
\UseMicrotypeSet[protrusion]{basicmath} % disable protrusion for tt fonts
}{}
\usepackage[margin=1in]{geometry}
\usepackage{hyperref}
\hypersetup{unicode=true,
            pdftitle={A comprehensive research of the social factors that affect homelessness in Los Angeles},
            pdfauthor={Xinyue Tang; Chen Wang},
            pdfborder={0 0 0},
            breaklinks=true}
\urlstyle{same}  % don't use monospace font for urls
\usepackage{graphicx,grffile}
\makeatletter
\def\maxwidth{\ifdim\Gin@nat@width>\linewidth\linewidth\else\Gin@nat@width\fi}
\def\maxheight{\ifdim\Gin@nat@height>\textheight\textheight\else\Gin@nat@height\fi}
\makeatother
% Scale images if necessary, so that they will not overflow the page
% margins by default, and it is still possible to overwrite the defaults
% using explicit options in \includegraphics[width, height, ...]{}
\setkeys{Gin}{width=\maxwidth,height=\maxheight,keepaspectratio}
\IfFileExists{parskip.sty}{%
\usepackage{parskip}
}{% else
\setlength{\parindent}{0pt}
\setlength{\parskip}{6pt plus 2pt minus 1pt}
}
\setlength{\emergencystretch}{3em}  % prevent overfull lines
\providecommand{\tightlist}{%
  \setlength{\itemsep}{0pt}\setlength{\parskip}{0pt}}
\setcounter{secnumdepth}{5}
% Redefines (sub)paragraphs to behave more like sections
\ifx\paragraph\undefined\else
\let\oldparagraph\paragraph
\renewcommand{\paragraph}[1]{\oldparagraph{#1}\mbox{}}
\fi
\ifx\subparagraph\undefined\else
\let\oldsubparagraph\subparagraph
\renewcommand{\subparagraph}[1]{\oldsubparagraph{#1}\mbox{}}
\fi

%%% Use protect on footnotes to avoid problems with footnotes in titles
\let\rmarkdownfootnote\footnote%
\def\footnote{\protect\rmarkdownfootnote}

%%% Change title format to be more compact
\usepackage{titling}

% Create subtitle command for use in maketitle
\providecommand{\subtitle}[1]{
  \posttitle{
    \begin{center}\large#1\end{center}
    }
}

\setlength{\droptitle}{-2em}

  \title{A comprehensive research of the social factors that affect homelessness
in Los Angeles}
    \pretitle{\vspace{\droptitle}\centering\huge}
  \posttitle{\par}
  \subtitle{An essay for Fall Data Challenge: Help Solve Homelessness}
  \author{Xinyue Tang\footnote{Undergraduate in Computer Science \& Engineering,
  Department of Computer Science, University of California, Davis.
  (\href{mailto:xiytang@ucdavis.edu}{\nolinkurl{xiytang@ucdavis.edu}})} \\ Chen Wang\footnote{Undergraduate in Computer Engineering, Samueli School
  of Engineering, University of California, Irvine.
  (\href{mailto:chenw23@uci.edu}{\nolinkurl{chenw23@uci.edu}})}}
    \preauthor{\centering\large\emph}
  \postauthor{\par}
      \predate{\centering\large\emph}
  \postdate{\par}
    \date{10/27/2019}


\begin{document}
\maketitle

{
\setcounter{tocdepth}{3}
\tableofcontents
}
\newpage

\hypertarget{overview-of-homeless-population-in-los-angeles}{%
\section{Overview of homeless population in Los
Angeles}\label{overview-of-homeless-population-in-los-angeles}}

\hypertarget{data-collection}{%
\subsection{Data Collection}\label{data-collection}}

First of all, we used the statistical data retrieved from U.S. Bureau of
Labor Statistics\footnote{U.S. Bureau of Labor Statistics. (2019,
  September). BLS Data Finder 1.1. Retrieved from
  \url{https://www.bls.gov/data/}.}, Zillow Group\footnote{Zillow Group.
  (2019, September). Houses For Rent in Los Angeles CA - 2,651 Homes
  \textbar{} Zillow. Retrieved from
  \url{https://www.zillow.com/los-angeles-ca/rent-houses/}.} and
California Department of Public Health\footnote{California Department of
  Public Health. (2019, September). CDPH Home. Retrieved from
  \url{https://www.cdph.ca.gov/}.} to provide a general impression of
trend and status quo of the homeless people and sheltered homeless
people in Los Angeles. The figure showing this perspective is in the
slides page 2 and 3.

\hypertarget{analysis-of-the-situation-of-the-homeless-population}{%
\subsection{Analysis of the situation of the homeless
population}\label{analysis-of-the-situation-of-the-homeless-population}}

From the figure, we can have a strong impression that in Los Angeles,
there are as many as ¾ of the homeless people that are not sheltered
yet. Furthermore, the overall homeless population is increasing
significantly in recent years. However, what disappoints us is that what
the government can provide for the homeless people - the sheltering- is
generally not increasing. They remain the same in the decade and
therefore making the contradiction of need for shelter and the providing
of shelter even more severe.

\hypertarget{analysis-of-the-factors-causing-homelessness}{%
\section{Analysis of the factors causing
homelessness}\label{analysis-of-the-factors-causing-homelessness}}

\hypertarget{factors-considered-and-data-source}{%
\subsection{Factors considered and data
source}\label{factors-considered-and-data-source}}

After having a knowledge of the general fact of homeless people in Los
Angeles, we have dug much deeper into the leading facts that make the
homeless so severe. We have searched and collected data from U.S. Bureau
of Labor Statistics\footnote{U.S. Bureau of Labor Statistics. (2019,
  September). BLS Data Finder 1.1. Retrieved from
  \url{https://www.bls.gov/data/}.} including the Consumption Price
Index(CPI), Minimum Wages, Unemployment rate, House rental price and
drug addict population in the city of Los Angeles.

\hypertarget{principal-component-analysis}{%
\subsection{Principal Component
Analysis}\label{principal-component-analysis}}

We want to examine whether these factors affect the results and how they
affect. Here we implement the Principal Component Analysis method to
process the data.

\hypertarget{processing-steps-for-principal-component-analysis}{%
\subsubsection{Processing steps for Principal Component
Analysis}\label{processing-steps-for-principal-component-analysis}}

For the first step, because the data value differ greatly in terms of
units and absolute value, we used Standardization into a range from 0 to
1 so that the different values will be able to compare with each other.
After standardization, we imported the data into R and utilized the
princomp function to perform the principal component analysis. From the
result, we can see that the first component has a Proportion of Variance
at 93\%, while the first two components have a Proportion of Variance of
97\%. This means the factors are in a highly uniform trend and can be
reduced to only one or two dimensions. After our looking more thoroughly
into the principal component analysis result, we will find out from the
loadings of the factors to the components, the five factors we are
observing are having almost equal relation with the dependent variable -
the overall homeless population. Therefore, we can conclude that these
factors possess a strong effect on the overall homeless people.

\hypertarget{reasoning-for-the-result-from-principal-component-analysis}{%
\subsubsection{Reasoning for the result from Principal Component
Analysis}\label{reasoning-for-the-result-from-principal-component-analysis}}

The increasing price index and house rental prices make living expenses
even higher for people with lower income and the rising unemployment
rate makes more people becoming homeless. What's more, drug addict, as a
common but severe social phenomenon, is also becoming more severe in Los
Angeles and we cannot deny the possibility that there are many people
fall into homelessness due to addiction to drugs.

\hypertarget{advice-proposed-to-the-government}{%
\section{Advice proposed to the
government}\label{advice-proposed-to-the-government}}

In conclusion, we are calling for the government to take various
measures to curb this phenomenon from various perspectives. They can
enhance the infrastructure construction to provide more shelter for the
homeless people to relieve this problem soon. They can also take
measures like banning or restricting drugs, providing more temporary
working opportunities for the unemployed people, controlling house
rental price or providing more alternative housing for low income
population and so on.


\end{document}
